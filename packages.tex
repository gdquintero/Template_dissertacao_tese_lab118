% Paquetes basicos cargados al inicio
\usepackage[utf8]{inputenc}
\usepackage{graphicx}
\usepackage{lmodern}
\usepackage[a4paper,top=2.54cm,bottom=2.0cm,left=2.0cm,right=2.54cm]{geometry} % margenes
\usepackage{amsmath,amsfonts,amssymb,amscd,amsthm,xspace}
\theoremstyle{plain}
\newtheorem{example}{Example}[chapter]
\newtheorem{theorem}{Theorem}[chapter]
\newtheorem{corollary}{Corollary}[chapter]
\newtheorem{lemma}{Lemma}[chapter]
\newtheorem{proposition}{Proposition}[chapter]
\newtheorem{axiom}{Axiom}[chapter]
\theoremstyle{definition}
\newtheorem{definition}{Definition}[chapter]
\newtheorem{assumption}{Assumption}
\renewcommand\theassumption{A\arabic{assumption}}

% Paquetes adicionales
\usepackage{natbib} % Bibliografia
\usepackage{lipsum} % Generar lero leros
\usepackage{setspace} % Para modificar el espaciio entre lineas
\setstretch{1.6} 
\usepackage{indentfirst} % Indentação do primeiro parágrafo

\usepackage{afterpage}

\newcommand\blankpage{%
    \null
    \thispagestyle{empty}%
    \addtocounter{page}{-1}%
    \newpage
}

\usepackage{multirow}
\usepackage{float}

\usepackage{url}

% Show line numbers
\usepackage[mathlines]{lineno}
\renewcommand\linenumberfont{\normalfont\small\color{gray}}

% Show labels
% \usepackage{showlabels}

\newcommand{\R}{\mathbb{R}}
\newcommand{\N}{\mathbb{N}}
\newcommand{\xtrial}{x^{\mathrm{trial}}}
\newcommand{\ytrial}{y^{\mathrm{trial}}}
\newcommand{\flow}{f_{\mathrm{low}}}
\newcommand{\Flow}{F_{\mathrm{low}}}
\newcommand{\Fmin}{F_{\mathrm{min}}}
\newcommand{\Imin}{I_{\mathrm{min}}}
\newcommand{\Imax}{I_{\mathrm{max}}}
\newcommand{\Pcal}{\mathcal{P}}

\DeclareMathOperator*{\Minimize}{Minimize}
\DeclareMathOperator*{\co}{co}
\DeclareMathOperator*{\inter}{int}

\usepackage{fancyhdr}
\usepackage{caption}
\usepackage{subcaption}

% %Estilo algoritmo Gustavo
\usepackage{enumitem} % Essencial pro estilo algoritmo gustavo (enumera os passos)
\setlist[enumerate,1]{itemindent=2em}
\setlist[enumerate,2]{itemindent=3em}
\newcounter{counter}
\counterwithin{counter}{chapter}
\newcommand{\algorithm}[3]{
\refstepcounter{counter}
\label{#1}
{\noindent\bf Algorithm~\thecounter:} #2
\vspace{-1.3\topsep}
\begin{enumerate}[leftmargin=2.5\parindent]
\renewcommand{\labelenumi}{\textbf{\theenumi}.}
\renewcommand{\labelenumii}{\textbf{\theenumii}.}

\renewcommand{\theenumi}{Step~\arabic{enumi}}
\renewcommand{\theenumii}{Step~\arabic{enumi}.\arabic{enumii}}

\setlength{\itemsep}{1ex}
\setlength{\parskip}{0pt}
\setlength{\parsep}{0pt}
#3
\end{enumerate}
}
\usepackage[usenames,svgnames,dvipsnames]{xcolor}% Para nuevos colores: DarkGreen, NavyBlue, DarkRed

\usepackage[pdftex,plainpages=false,pdfpagelabels,pagebackref,colorlinks=true,citecolor=DarkGreen,linkcolor=NavyBlue,urlcolor=DarkRed,filecolor=green,bookmarksopen=true]{hyperref} % links coloridos
