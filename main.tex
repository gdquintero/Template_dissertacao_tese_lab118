\documentclass[11pt,oneside,a4paper]{book}
% Paquetes basicos cargados al inicio
\usepackage[utf8]{inputenc}
\usepackage{graphicx}
\usepackage{lmodern}
\usepackage[a4paper,top=2.54cm,bottom=2.0cm,left=2.0cm,right=2.54cm]{geometry} % margenes
\usepackage{amsmath,amsfonts,amssymb,amscd,amsthm,xspace}
\theoremstyle{plain}
\newtheorem{example}{Example}[chapter]
\newtheorem{theorem}{Theorem}[chapter]
\newtheorem{corollary}{Corollary}[chapter]
\newtheorem{lemma}{Lemma}[chapter]
\newtheorem{proposition}{Proposition}[chapter]
\newtheorem{axiom}{Axiom}[chapter]
\theoremstyle{definition}
\newtheorem{definition}{Definition}[chapter]
\newtheorem{assumption}{Assumption}
\renewcommand\theassumption{A\arabic{assumption}}

% Paquetes adicionales
\usepackage{natbib} % Bibliografia
\usepackage{lipsum} % Generar lero leros
\usepackage{setspace} % Para modificar el espaciio entre lineas
\setstretch{1.6} 
\usepackage{indentfirst} % Indentação do primeiro parágrafo

\usepackage{afterpage}

\newcommand\blankpage{%
    \null
    \thispagestyle{empty}%
    \addtocounter{page}{-1}%
    \newpage
}

\usepackage{multirow}
\usepackage{float}

\usepackage{url}

% Show line numbers
\usepackage[mathlines]{lineno}
\renewcommand\linenumberfont{\normalfont\small\color{gray}}

% Show labels
% \usepackage{showlabels}

\newcommand{\R}{\mathbb{R}}
\newcommand{\N}{\mathbb{N}}
\newcommand{\xtrial}{x^{\mathrm{trial}}}
\newcommand{\ytrial}{y^{\mathrm{trial}}}
\newcommand{\flow}{f_{\mathrm{low}}}
\newcommand{\Flow}{F_{\mathrm{low}}}
\newcommand{\Fmin}{F_{\mathrm{min}}}
\newcommand{\Imin}{I_{\mathrm{min}}}
\newcommand{\Imax}{I_{\mathrm{max}}}
\newcommand{\Pcal}{\mathcal{P}}

\DeclareMathOperator*{\Minimize}{Minimize}
\DeclareMathOperator*{\co}{co}
\DeclareMathOperator*{\inter}{int}

\usepackage{fancyhdr}
\usepackage{caption}
\usepackage{subcaption}

% %Estilo algoritmo Gustavo
\usepackage{enumitem} % Essencial pro estilo algoritmo gustavo (enumera os passos)
\setlist[enumerate,1]{itemindent=2em}
\setlist[enumerate,2]{itemindent=3em}
\newcounter{counter}
\counterwithin{counter}{chapter}
\newcommand{\algorithm}[3]{
\refstepcounter{counter}
\label{#1}
{\noindent\bf Algorithm~\thecounter:} #2
\vspace{-1.3\topsep}
\begin{enumerate}[leftmargin=2.5\parindent]
\renewcommand{\labelenumi}{\textbf{\theenumi}.}
\renewcommand{\labelenumii}{\textbf{\theenumii}.}

\renewcommand{\theenumi}{Step~\arabic{enumi}}
\renewcommand{\theenumii}{Step~\arabic{enumi}.\arabic{enumii}}

\setlength{\itemsep}{1ex}
\setlength{\parskip}{0pt}
\setlength{\parsep}{0pt}
#3
\end{enumerate}
}
\usepackage[usenames,svgnames,dvipsnames]{xcolor}% Para nuevos colores: DarkGreen, NavyBlue, DarkRed

\usepackage[pdftex,plainpages=false,pdfpagelabels,pagebackref,colorlinks=true,citecolor=DarkGreen,linkcolor=NavyBlue,urlcolor=DarkRed,filecolor=green,bookmarksopen=true]{hyperref} % links coloridos

\newcommand{\autor}{Name of the work's author}
\newcommand{\tituloen}{English title}
\newcommand{\titulopt}{Título em Portugués}
\newcommand{\orientador}{Prof. Dr. Advisor name}
\newcommand{\coorientador}{Prof. Dr. Co-advisor name}

\begin{document}
\fancyhead{} % Clears all page headers and footers
\rhead{\thepage} % Sets the right side header to show the page number
\lhead{} % Clears the left side page header

\pagestyle{fancy} % The page style headers have been "empty" all this time, now use the "fancy" headers as defined before to bring them back

\renewcommand{\backrefpagesname}{Cited in page(s):~}
% Texto padrão antes do número das páginas
\renewcommand{\backref}{}
% Define os textos da citacao
\renewcommand*{\backrefalt}[4]{
	\ifcase #1 %
		No citations in the text.%
	\or
		Cited on page #2.%
	\else
		Cited #1 times on pages #2.%
	\fi}%
% ---
% TIPO DE TRABALHOS
% Copiar e colar na parte "Tipo de trabalho" da capa

% Qualificação:

%RELATÓRIO APRESENTADO AO\\
%INSTITUTO DE MATEMÁTICA E ESTATÍSTICA\\
%DA UNIVERSIDADE DE SÃO PAULO\\
%PARA EXAME DE QUALIFICAÇÃO DE\\
%DOUTOR(A) EM CIÊNCIAS

% Dissertação:

%DISSERTAÇÃO APRESENTADA AO\\
%INSTITUTO DE MATEMÁTICA E ESTATÍSTICA\\
%DA UNIVERSIDADE DE SÃO PAULO\\
%PARA OBTENÇÃO DO TÍTULO DE\\
%MESTRE(A) EM CIÊNCIAS

% Tese:

%TESE APRESENTADA AO\\
%INSTITUTO DE MATEMÁTICA E ESTATÍSTICA\\
%DA UNIVERSIDADE DE SÃO PAULO\\
%PARA OBTENÇÃO DO TÍTULO DE\\
%DOUTOR(A) EM CIÊNCIAS

% APOIO
% Copiar e colar na parte "Tipo de apoio" da capa
% Norma sobre agradecimento por auxílios da FAPESP:
% https://fapesp.br/11789/referencia-ao-apoio-da-fapesp-em-todas-as-		formas-de-divulgacao
%
% Norma sobre agradecimento por auxílios da CAPES (Portaria 206,
% de 4 de Setembro de 2018):
% https://www.in.gov.br/materia/-/asset_publisher/Kujrw0TZC2Mb/content/id/39729251/do1-2018-09-05-portaria-n-206-de-4-de-setembro-de-2018-39729135

% Apoio (CAPES):

% O presente trabalho foi realizado com apoio da Coordenação
%       de Aperfeiçoamento\\ de Pessoal de Nível Superior -- Brasil
%       (CAPES) -- Código de Financiamento 001}, % o código é sempre 001
%
% This study was financed in part by the Coordenação de
%       Aperfeiçoamento\\ de Pessoal de Nível Superior -- Brasil
%       (CAPES) -- Finance Code 001, % o código é sempre 001

% Apoio (FAPESP):

% Durante o desenvolvimento deste trabalho, o autor recebeu\\
%       auxílio financeiro da FAPESP -- processo nº aaaa/nnnnn-d,
%
% During the development if this work, the author received\\
%       financial support from FAPESP -- grant \#aaaa/nnnnn-d,

% Capa
\begin{titlepage}
\begin{center}
\begin{minipage}[t][56mm][s]{96mm}
          \vspace*{2cm plus 1.5cm minus 1.8cm}

          \centering

          {\Large \bfseries\tituloen}\\[0.4cm] % Thesis title

          \vspace{1cm plus 1cm minus 0.6cm}

          {\Large\autor}

          \vspace*{2cm plus 1.5cm minus 1.8cm}
      \end{minipage}

\vfill
% Dissertação:


% Tipo de trabalho
    \textsc{\large{
    REPORT PRESENTED TO THE\\
    INSTITUTE OF MATHEMATICS AND STATISTICS\\
    OF THE UNIVERSITY OF SÃO PAULO\\
    FOR THE DOCTOR OF SCIENCE\\
    QUALIFYING EXAMINATION}}
    
    \vskip 1.5cm
    \large{
    Program: Applied Mathematics\\
    Advisor: \orientador\\
    Co-advisor: \coorientador}

   	\vfill
   	% Apoio
    \normalsize{This study was financed in part by the Coordenação de
         Aperfeiçoamento\\ de Pessoal de Nível Superior -- Brasil
         (CAPES) -- Finance Code 001}
    
    \vskip 1.5cm
    \normalsize{São Paulo, Brasil}\\
    \normalsize{August 2022}
\end{center}
\end{titlepage}

\newpage
% Para a folha de rosto copiar segundo o tipo de trabalho na parte "Texto folha de rosto" abaixo.

% Qualificação:

%This is the original version of the qualifying text
%prepared by candidate \autor,
%as submitted to the Examining Committee.

%Esta é a versão original do texto de qualificação elaborado pelo(a) candidato(a) \autor, 
%tal como submetido à Comissão Julgadora.

% Dissertação:

%This is the original version of the master thesis prepared
%by candidate \autor, as submitted to the Examining Committee.

%Esta é a versão original da dissertação elaborada pelo(a) candidato(a) \autor, tal como submetida à Comissão Julgadora.

% Tese:

%This is the original version of the thesis prepared
%by candidate \autor, as submitted to the Examining Committee.

%Esta é a versão original da tese elaborada pelo(a) candidato(a) \autor, tal como submetida à Comissão Julgadora.

%Folha de rosto
\thispagestyle{empty}
    \begin{center}
 \begin{minipage}[t][56mm][s]{96mm}
          \vspace*{2cm plus 1.5cm minus 1.8cm}
          \centering
          {\Large \bfseries \tituloen}\\[0.4cm]
           \vspace{1cm plus 1cm minus 0.6cm}

           {\Large\autor}

          \vspace*{2cm plus 1.5cm minus 1.8cm}
      \end{minipage}
     \end{center}
\vskip 2cm
\setstretch{1.3}
   \begin{flushright}
      \begin{minipage}[t][50mm][s]{80mm}
        \begin{flushright}
        % Texto folha de rosto
          \normalsize{
          This is the original version of the qualifying text
          prepared by candidate \autor,
          as submitted to the Examining Committee. 
          }
        \end{flushright}
        \vspace*{0pt plus 50mm}
      \end{minipage}
      \par
    %} % fbox
  \end{flushright}

\afterpage{\blankpage}

% Agradecimentos
\clearpage 
\thispagestyle{plain}
\pagenumbering{roman} 
\begin{center}{\huge{\textit{Acknowledgements}} \par}\end{center}
\vspace{1em}
\lipsum[1]
\lipsum[1]
\vfil\vfil\null

% Resumo
\afterpage{\blankpage}
\clearpage % Start a new page
 \thispagestyle{empty}
  %\null\vfil
  \begin{flushleft}
    \setlength{\parskip}{0pt}
    {\centering{\huge{\textit{Resumo}}} \par} 
    \bigskip
		\begin{center}
\noindent\begin{minipage}{0.75\textwidth}
\small{\autor. {\bfseries\titulopt}. 
Exame de Qualificação (Doutorado) - Instituto de Matemática e Estatística,
Universidade de São Paulo, São Paulo, 2022.}
\end{minipage}
\end{center}
\end{flushleft}
\bigskip
Bla bla bla bla bla bla bla bla bla bla bla bla bla bla bla bla bla bla bla bla bla bla bla bla bla bla bla bla bla bla bla bla bla bla bla bla bla bla bla bla bla bla bla bla bla bla bla bla bla bla bla bla bla bla bla bla bla bla bla bla bla bla bla bla bla bla bla bla bla bla bla bla bla bla bla bla bla bla bla bla bla bla bla bla bla bla bla bla bla bla bla bla bla bla bla bla bla bla bla bla bla bla bla bla bla bla bla bla bla bla bla bla bla bla bla bla bla bla bla bla bla bla bla bla bla bla bla bla bla bla bla bla bla bla bla bla bla bla bla bla bla bla bla bla bla bla bla bla bla bla bla bla bla bla bla bla bla bla bla bla bla bla bla\par
\vspace{0.5cm}
\noindent\textbf{Palavras-chave:} Keyword1, Keyword2, Keyword3.

% Abstract
\afterpage{\blankpage}
\clearpage % Start a new page
\thispagestyle{empty}
  %\null\vfil
  \begin{flushleft}
    \setlength{\parskip}{0pt}
    {\centering{\huge{\textit{Abstract}}} \par} 
    \bigskip
		\begin{center}
\noindent\begin{minipage}{0.75\textwidth}
\small{\autor. \textbf\tituloen. 
Qualifying Exam (Doctorate) - Institute of Matemathics and Statistics,
University of São Paulo, São Paulo, 2022.}
\end{minipage}
\end{center}
\end{flushleft}
\bigskip
Bla bla bla bla bla bla bla bla bla bla bla bla bla bla bla bla bla bla bla bla bla bla bla bla bla bla bla bla bla bla bla bla bla bla bla bla bla bla bla bla bla bla bla bla bla bla bla bla bla bla bla bla bla bla bla bla bla bla bla bla bla bla bla bla bla bla bla bla bla bla bla bla bla bla bla bla bla bla bla bla bla bla bla bla bla bla bla bla bla bla bla bla bla bla bla bla bla bla bla bla bla bla bla bla bla bla bla bla bla bla bla bla bla bla bla bla bla bla bla bla bla bla bla bla bla bla bla bla bla bla bla bla bla bla bla bla bla bla bla bla bla bla bla bla bla bla bla bla bla bla bla bla bla bla bla bla bla bla bla bla bla bla bla\par
\vspace{0.5cm}
\noindent\textbf{Keyword:} Keyword1, Keyword2, Keyword3.

% Sumario
\lhead{\emph{Contents}} % Set the left side page header to "Contents"
\let\cleardoublepage\clearpage
\tableofcontents

% Lista de figuras
\lhead{\emph{List of Figures}} % Set the left side page header to "List of Figures"
\listoffigures % Write out the List of Figures
\addcontentsline{toc}{chapter}{List of Figures}

% Lista de tabelas
\lhead{\emph{List of Tables}} % Set the left side page header to "List of Tables"
\listoftables % Write out the List of Tables
\addcontentsline{toc}{chapter}{List of Tables}

\setstretch{1.5} % Set the line spacing to 1.5, this makes the following tables easier to read

% Lista de abreviacoes
\lhead{\emph{Abbreviations}} % Set the left side page header to "Abbreviations"
\chapter*{Abbreviations}
\addcontentsline{toc}{chapter}{Abbreviations}

% Lista de simbolos
\lhead{\emph{Symbols}} % Set the left side page header to "Symbols"
\chapter*{List of Symbols}
\begin{tabular}{ll}
        $\omega$    & Frequência angular\\
        $\psi$      & Função de análise \emph{wavelet}\\
        $\Psi$      & Transformada de Fourier de $\psi$\\
\end{tabular}
\addcontentsline{toc}{chapter}{List of Symbols}

% Dedicatoria
\clearpage
\thispagestyle{empty}
\setstretch{1.3} % Return the line spacing back to 1.3
\pagestyle{empty} % Page style needs to be empty for this page
\vspace*{\fill}
\begin{center}
{\Large\it to my parents}
\end{center}
\vspace*{\fill}

\mainmatter % Begin numeric (1,2,3...) page numbering
\pagestyle{fancy}
\setlength{\parskip}{0.5\baselineskip} % Set space between paragraphs
% Conteudo
\linenumbers
% Chapter Template

\chapter{Introduction} % Main chapter title

\label{Chapter1}

\lhead{Chapter 1. \emph{Introduction}} % Change X to a consecutive number; this is for the header on each page - perhaps a shortened title
\lipsum[1]\\
\lipsum[1]\\
\lipsum[1]

Figure~\ref{fig:ovexamples} below represents each of the situations given in the previous examples, for different values of $J$. Given 4 functions (a), the black functions in (b), (c) and (d), represent the Order-Value functions for the Min-min, Min-max and VaR-like problems, respectively. 
\begin{figure}[H]
     \centering
     \begin{subfigure}{0.32\textwidth}
         \centering
         \includegraphics[width=\textwidth]{pictures/scenarios.pdf}
         \caption{$m=4$}
         \label{fig:scenarios}
     \end{subfigure}
     \hfill
     \begin{subfigure}{0.32\textwidth}
         \centering
         \includegraphics[width=\textwidth]{pictures/minimin.pdf}
         \caption{$J=\{1\}$}
         \label{fig:minimin}
     \end{subfigure}
     \hfill
     \begin{subfigure}{0.32\textwidth}
         \centering
         \includegraphics[width=\textwidth]{pictures/minimax.pdf}
         \caption{$J=\{4\}$}
         \label{fig:minimax}
     \end{subfigure}

     \begin{subfigure}{0.32\textwidth}
         \centering
         \includegraphics[width=\textwidth]{pictures/var.pdf}
         \caption{$J=\{2\}$}
         \label{fig:var}
     \end{subfigure}
     \hfill
     \begin{subfigure}{0.32\textwidth}
         \centering
         \includegraphics[width=\textwidth]{pictures/lovo.pdf}
         \caption{$J=\{1,2\}$}
         \label{fig:lovo}
     \end{subfigure}
     \hfill
     \begin{subfigure}{0.32\textwidth}
         \centering
         \includegraphics[width=\textwidth]{pictures/cvar.pdf}
         \caption{$J=\{3,4\}$}
         \label{fig:cvar}
     \end{subfigure}
        \caption{Order-Value functions for $J\subset\{1,2,3,4\}$.}
        \label{fig:ovexamples}
\end{figure}

 
% Chapter Template

\chapter{Chapter title}\label{Chapter2}
\lhead{Chapter 2. \emph{Order-Value Optimization}} 
\lipsum
\newpage
\section{Example of an algorithm}
\algorithm{alg:uvar}{Let $\delta>0$, $\sigma_{\min}>0$,
$\alpha\in(0,1)$, and $x^0 \in \R^n$ be given. Set $k\leftarrow 0$.}{
    \item\label{alg:uvar-step1} Choose $\sigma \geq \sigma_{\min}$ and initialize
    $\ell\leftarrow 1$.
    \begin{enumerate}
        \item\lipsum[1]
    \end{enumerate}
    \item\label{alg:uvar-step2} Compute $\xtrial$ as a solution of
    \begin{equation} \label{uvar-subpro}
      \Minimize \max_{i \in I(x^k,\delta)}
      \left\{ \nabla f_i(x^k)^T (x - x^k)\right\} +  \dfrac{\sigma}{2}\|x - x^k\|^2 .
    \end{equation}
}

 
\nolinenumbers

% Bibliografia
\backmatter
\nocite{*}
\lhead{\emph{Bibliography}}
\addcontentsline{toc}{chapter}{Bibliography}
\bibliographystyle{apalike}
\bibliography{bibliography}
\end{document}